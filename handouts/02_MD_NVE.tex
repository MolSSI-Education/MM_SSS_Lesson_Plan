%\documentclass[12pt]{article}
\documentclass[aip,jcp,preprint,superscriptaddress,floatfix]{revtex4-1}
\usepackage{url,graphicx,tabularx,array,geometry,amsmath,listings}
\setlength{\parskip}{2ex} %--skip lines between paragraphs
\setlength{\parindent}{20pt} %--don't indent paragraphs

\setlength{\headheight}{-50pt}
\setlength{\textheight}{700pt}
\setlength{\textwidth}{500pt}
\setlength{\oddsidemargin}{-10pt}
\setlength{\footskip}{50pt}
\usepackage{graphicx}% Include figure files
\usepackage{bm}
\usepackage{hyperref}
\graphicspath{{./Figures/}}

%-- Commands for header
\renewcommand{\title}[1]{\textbf{\large{#1}}\\}
\renewcommand{\line}{\begin{tabularx}{\textwidth}{X>{\raggedleft}X}\hline\\\end{tabularx}\\[-0.5cm]}
\newcommand{\leftright}[2]{\begin{tabularx}{\textwidth}{X>{\raggedleft}X}#1%
& #2\\\end{tabularx}\\[-1cm]}

%\linespread{2} %-- Uncomment for Double Space
\begin{document}

\title{\center{Molecular Dynamics} }
\rule{\textwidth}{1pt}
\leftright{The Molecular Sciences Software Institute}{Eliseo Marin-Rimoldi} %-- left and right positions in the header

\bigskip

\section{Introduction}

Your first assignment consisted on implementing a Monte Carlo (MC) simulation
for the Lennard-Jones (LJ) particle in the canonical ensemble. 
As part of this task,
you developed code that computed pair-wise energies and forces for this model, 
as well as functions that computed interaction energies between molecules,
the total system energy and the system virial.

The objective of the new task is to extend your code by including the
necessary functionality to simulate the NVE ensemble of configurations 
of the LJ model using the Molecular Dynamics (MD). You will be 
assigned to work in teams to develop this project. The team will
decide how is best to design the new code: what is the shared functionality
between the two methods? how does the library should look like? How can
we improve reusability and extensibility? what data
structures should be used? What are the unit tests 
that might be used? The primary learning goal of this activity 
is to think about interoperability and teamwork.

In the following sections, the relevant equations of the MD 
method will be presented and reference data for validation will be provided.

\subsection{Molecular Dynamics method}

A very common technique to simulate equilibrium and nonequilibrium
properties is the Molecular Dynamics method. 
The objective of this method is to simulate the evolution of a model over time
by integrating the Newtonian equations of motion to 
extract dynamic and thermodynamic properties. Thermodynamic
averages should be identical to those obtained by MC sampling
due to the ergodic hypothesis, given that we use the same model at the
same conditions. In an MD simulation, we can get additional dynamical
properties that is inaccessible in an MC simulation, such diffusivities or 
viscosities. 

Given a conservative system (i.e. the Hamiltonian does not depend on time), 
we can write the Newtonian equations of motion in Cartesian coordinates
from Hamilton's
equations as

\begin{equation}
	\ddot{\mathbf{r}} = \frac{\mathbf{f}_i\left( \mathbf{r} \right) }{\textit{m}_i}
\end{equation}

Where $\mathbf{r}$ is the vector of Cartesian coordinates, $\mathbf{f}_i$ and
 $\textit{m}_i$ are the forces and the mass of each atom, respectively.
These are 3N second order differential equations, where N is the number 
of atoms in the system. There are various finite difference approaches to solve
such equations. The basic idea in all of them is to move 
across time using finite time steps $\delta t$ and thus generate new atomic
positions. A good integration scheme is one that conserves energy 
and momentum, minimizes force evaluations, is stable and accurate. 

\section{Velocity Verlet Algorithm}

One particularly good and widely used method is the Velocity Verlet algorithm
which is an improvement on the original Verlet method (i.e. less memory
intensive). The equations of this method are

\begin{equation}
	r \left( t + \delta t \right) = r \left( t \right) + v \left( t \right) \delta t + \frac{f \left(t \right) }{2 m} \delta t ^2
\label{eq:verlet.position.update}
\end{equation}

\begin{equation}
	v \left( t + \delta t \right) = v \left( t \right) + \frac{f \left( t + \delta t \right) + f \left(t \right) }{2m} \delta t
\end{equation}

This algorithm can be implemented as follows

\begin{itemize}
	\item Given a system state (i.e. positions and velocities), 
		compute forces on each atom
	\item Update the positions using Equation \ref{eq:verlet.position.update}
	\item Do a partial update to the velocities using the current forces
		\begin{equation}
			v \leftarrow v + \frac{f}{2m} \delta t
		\end{equation}
	\item Compute new forces using the updated positions
	\item Complete the update of the velocities with the old forces
		\begin{equation}
			v \leftarrow v + \frac{f}{2m} \delta t
		\end{equation}

\end{itemize}

\subsection{Technical considerations}

\textbf{Energy conservation. } The velocity verlet algorithm has very good
energy conservation properties. It is always a good check to see if 
there is no drift in energy during the course of the simulation.

\textbf{Time step. } Using too large of a time step will result in an unstable
system, with energy increasing in time. 

\textbf{Initial positions. } It is possible to use a crystaline 
lattice as an initial
configuration. You could also place atoms randomly, but this might be
dangerous as core overlaps might exist, leading to huge repulsion forces. In
this case, you could first relax the system using the translation moves
of your MC code, which only use energy differences forces for sampling. 
Finally, you could use an old initial configuration 
generated from your MC code.

\textbf{Initial velocities. } The initial velocities are usually obtained
using the Maxwell-Boltzmann distribution at the desired temperature

\begin{equation}
	\rho \left(\mathbf{v}_i\right) = \left( \frac{m_i}{2 \pi k_B T} \right) ^ {1/2} exp \left( - \frac{m_i \mathbf{v}_i^2 } {2 k_B T }  \right)
\end{equation}

Note that this is a Gaussian distribution with zero mean and variance

\begin{equation}
	\sigma_v^2 = \frac{k_B T}{m_i}
\end{equation}

Once the velocities are assigned to the atoms, we must set the system momentum
to zero to avoid any translational drift. We do this by 

\begin{itemize}
	\item Find the net momentum of the system $\mathbf{P} = \sum m_i \mathbf{v}_i$
	\item Reassign initial atomic velocities as
		\begin{equation}
			\mathbf{v}_i \leftarrow \mathbf{v}_i - \frac{\mathbf{P}} { N m_i}
		\end{equation}
\end{itemize}

\textbf{Periodic boundary conditions, minimum image distance, 
energy truncation and tail correction. } Your MD code should also implement 
these features.

\end{document}
